\section{Data Quality}
\label{sec:DataQuality}

\subsection{Terror Attacks Dataset}
Multiple issues regarding data quality have been found in this dataset:

\paragraph{Broadness} 
The dataset comprises attacks ranging from 1969 to 1950 and spanning the entire globe. Simple and relevant explanations for the graph formation or properties are not likely to be found, since the mechanisms behind two different attacks can be entirely different.

\paragraph{Structure} 
Half of the nodes are isolated, hence the topological information they carry in the graph is very limited. What is more, because of the transitivity relation described in Section~\ref{subsec:Terror Attacks Dataset}, connected components are in most of the cases complete, hence isotropic. 

\paragraph{Reliability} 
Errors have been found in the data. For example nodes
 \texttt{Djibouti\_Youth\_Movement\_19900927} 
 and 
 \texttt{Armed\_Islamic\_Group\_19950711} 
 have been connected, whereas the first attack took place in Djibouti~\cite{amnesty1991} and the second one in Paris~\cite{nouvelObs2007}. Hence algorithms using the data must tolerate some error in order to avoid overfitting.
