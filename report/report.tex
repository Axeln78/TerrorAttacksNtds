% XeLaTeX can use any Mac OS X font. See the setromanfont command below.
% Input to XeLaTeX is full Unicode, so Unicode characters can be typed directly into the source.

% The next lines tell TeXShop to typeset with xelatex, and to open and save the source with Unicode encoding.

%!TEX TS-program = xelatex
%!TEX encoding = UTF-8 Unicode

\documentclass[12pt]{article}
\usepackage{geometry}
 \geometry{
 a4paper,
 left=20mm,
 right=20mm,
 top=20mm,
 bottom=20mm
 } 
%\geometry{landscape}                % Activate for for rotated page geometry
%\usepackage[parfill]{parskip}    % Activate to begin paragraphs with an empty line rather than an indent
\usepackage{graphicx}
\usepackage{amssymb}
\usepackage{amsmath}
\usepackage{float}
\usepackage{siunitx}
\usepackage{booktabs}
\usepackage[textfont={it}]{caption}
\usepackage{subcaption}
\usepackage[htt]{hyphenat}
\usepackage[]{algorithm2e}
\SetAlCapNameFnt{\it} 
\SetAlCapFnt{\rm}
\graphicspath{{../notebooks/pics/}}
% Will Robertson's fontspec.sty can be used to simplify font choices.
% To experiment, open /Applications/Font Book to examine the fonts provided on Mac OS X,
% and change "Hoefler Text" to any of these choices.

\def\labelitemi{--}

\usepackage{fontspec,xltxtra,xunicode}
\defaultfontfeatures{Mapping=tex-text}
\setromanfont[Mapping=tex-text]{Palatino}
\setsansfont[Scale=MatchLowercase,Mapping=tex-text]{Gill Sans}
\setmonofont[Scale=MatchLowercase]{Andale Mono}
\setlength\parindent{0pt}

\title{\vspace{-4em}\noindent\rule{\textwidth}{.5pt}\\\emph{Predicting Terror Attacks?}\\A Data Story\\
\vspace{-.6em}\noindent\rule{\textwidth}{.5pt}}
\author{Axel Nilsson, Nicolas Bollier, Elias Le Boudec, Enea Figini\\Team 29}
%\date{}                                           % Activate to display a given date or no date
\makeatletter
\renewcommand{\@algocf@capt@plain}{above}% formerly {bottom}
\makeatother

\begin{document}
\maketitle
\pagebreak
% For many users, the previous commands will be enough.
% If you want to directly input Unicode, add an Input Menu or Keyboard to the menu bar 
% using the International Panel in System Preferences.
% Unicode must be typeset using a font containing the appropriate characters.
% Remove the comment signs below for examples.

% \newfontfamily{\A}{Geeza Pro}
% \newfontfamily{\H}[Scale=0.9]{Lucida Grande}
% \newfontfamily{\J}[Scale=0.85]{Osaka}

\section{Introduction}
\label{sec:Introduction}
Exploring the dataset ``Terror Attacks" led to formulating the following question: is it possible to predict the location of a terrorist attack given a list of features of this attack? 

The goal of this project is to answer this question using data analysis tools provided by the course ``\textit{A Network Tour Of Data Science}".
\begin{frame}
\frametitle{Are terrorist relationships similar to social networks?}

If we find that relationships networks are similar to social networks then we could eventually understand how they form and how people join terrorist organisations.


The relationships line graph is built from a graph and we can only get partial information about this original graph.

The $transitivity$ and $homophily$ particularities of social networks gives us guidelines for building a graph with similar proprieties as social structures. 
In social networks, it is more likely to connect components between themselves rather than extending chains.

We found that some social networks are scale-free. 
We can only use one connected component, so we use the largest one of the dataset.

\end{frame}

% ----------------------------------------------------------------------------------------

\begin{frame}
\frametitle{Making a comparable line graph}
We build the line graph from a scale free graph that has a node number $n$.

\begin{figure}[H]
\begin{center}
    \begin{subfigure}[b]{0.4\textwidth}
        \includegraphics[width=\textwidth]{graphScaleFree.png}
        \caption{Scale free network}
        \label{fig:Scalefree}
    \end{subfigure}
    \begin{subfigure}[b]{0.4\textwidth}
        \includegraphics[width=\textwidth]{graphLineScaleFree.png}
        \caption{Line graph of the scale free network}
        \label{fig:lineG}
    \end{subfigure}
\label{fig:RelationshipScaleFree}
\end{center}
\end{figure}

\end{frame}

% ----------------------------------------------------------------------------------------

\begin{frame}
\frametitle{Relationships dataset: Results}

\begin{figure}[H]
\begin{center}
\includegraphics[width=.5\textwidth]{DegreeDiff.png}
\caption{Difference of degree distribution between the dataset and the generated line graph}
\label{fig:degdiff}
\end{center}
\end{figure}

Preliminary conclusion: The relationship network cannot be modeled by the line graph of a scale free network
\begin{itemize}
\item This could be because the relations of terrorist are not similar to social ties
\item Possibly because the size of the largest component is too small, making an unrealistically small number of relationships for the scale free graph to represent correctly a social network.
\end{itemize}

\end{frame}

% ----------------------------------------------------------------------------------------
\section{Terrorist Relationships as a Social Network}
\label{sec:Terrorist Relationships as a Social Network}
In this section, the properties of the terrorist relationships line graph as a relational network is explored. As~\cite{ZSG2006} mentions, an organisation needs interpersonal connection to function and studying the structure of the social organisation could allow to predict the evolution of terrorist organisations.

\cite{krawczyk_line_2011}~found that on the basis of a study of an online social network, such a network could be well approximated by the line graph of a scale free network. Assuming that there exists a transitivity relation between nodes (see Equation~\ref{eq:transitivity}) as one would expect in a social network, a scale free network corresponding to the relationships graph has been generated. Its line graph counterpart has then been compared to the relationships graph (see Figure~\ref{fig:RelationshipScaleFree} below).
%If that propriety can be verified by our dataset, %then we could gain information from the original graph from which the line graph originates.
%
%Social sciences studies have shown that social/relationship networks have the particularities of homophily and transitivity.
%Logically if $a$ \& $b$ are friends and $c$\& $b$ are also, the it is more likely that $a$ \& $c$ are friends than not. This mathematically translates to:
%\begin{equation}
%	a \sim b \text{ and } b \sim c \text{ then } a \sim c
%\end{equation}
%
%As a first research question we will try to verify that our dataset derives from a scale-free network, implying that the graph that generated the relationship dataset have proprieties similar to social networks.
%By creating a scale-free network and making its line graph, we compared the degree distribution of the relationship dataset we were able to show that.

\begin{figure}[H]
\begin{center}
    \begin{subfigure}[b]{0.4\textwidth}
        \includegraphics[width=\textwidth]{graphScaleFree.png}
        \caption{Scale free network}
        \label{fig:Scalefree}
    \end{subfigure}
    ~
    \begin{subfigure}[b]{0.4\textwidth}
        \includegraphics[width=\textwidth]{graphLineScaleFree.png}
        \caption{Line graph of scale free network}
        \label{fig:lineG}
    \end{subfigure}
    
    \begin{subfigure}[b]{\textwidth}
    	\begin{centering}
        \includegraphics[width=.5\textwidth]{DegreeDiff.png}
        \caption{\centering Comparison of the degree distribution of the two line graphs.}
        \label{fig:DegDiff}
        \end{centering}
    \end{subfigure}
\caption{Comparison of a scale free network and the terrorist relationships network.}
\label{fig:RelationshipScaleFree}
\end{center}
\end{figure}

From the degree distribution comparison, we can conclude that the terrorist relationships network shows no significant similarity with a  social network. 
%\subsection{Terrorist Relations Dataset}
%
%\subsection{Terror Attacks Dataset}
%\label{subsec:terror attack quality}

\section{Predictions}
\label{sec:Predictions}
The algorithm used to predict the terror attack location is the following:
\begin{enumerate}
\item From the dataset, select the 10 biggest connected components (``component'' in what follows).
\item Sort the dataset by date of terror attack.
\item At this point, a component represents a location, and the nodes are the terror attacks in chronological order.
\item Select one node per component that is strongly connected to the others, the ``lead" node.
\item Find the lead node $l^\star$ that is the most strongly linked to the new node (i.e. the next terror attack).
\item The predicted location of the next terror attack is the location of the component $l^\star$ belongs to.
\end{enumerate}

The determination of the lead node uses the features vector supplied with each node, and a weighting function $w$.
Let $w$ be the application that returns a weight for each pair of nodes $(n_1,n_2)$ in the graph $\mathcal{G}$, defined as
\begin{align}
w:~ \mathcal{G}^2	& \to \mathbb{R}^+ \\
(n_1,n_2) 			& \mapsto f(|n_1-n_2|)
\end{align}
where
\begin{equation}
|n_1-n_2| = \| \text{features}(n_1)-\text{features}(n_2)\|_2 \\
\end{equation}
$\text{features}(n)$ is a binary features vector for each node $n\text{ in }\mathcal{G}$ and $f:~\mathbb{R}^+ \to  \mathbb{R}^+$ is the node distance weighting.
Examples for $f$ are given in Table~\ref{tab:Prediction accuracy for different node distance weightings}.

For each connected component, the lead node is determined as described below.

\begin{algorithm}[H]
\vspace{-.5em}

 \KwData{Connected component $C$}
 
 \KwResult{Lead node $n_l$}
 
 Initialise $s(n)$ to zero. $s$ is a dictionary mapping a score $s(n)$ for each node $n$
 
 \For{each edge $e$ from $C$}{

Let $e=(n_1,n_2)$, $w$ be the weight of $e$

$s(n_1) \leftarrow s(n_1)+w$

$s(n_2) \leftarrow s(n_2)+w$

}
\KwRet{$n_l = \arg\max_{n \in C}s(n)$\vspace{.5em}}
 \label{alg:leadNode}
 \caption{Finding the lead node of a connected component with weighted edges}
\end{algorithm}

Finally, the prediction algorithm is presented below. 

\begin{algorithm}[H]
\vspace{-.5em}

 \KwData{Set of connected components $\{C_i^t\}$, $i=1,\dots,10$, and
 the features vector of the next terror attack $n_{t+1}$, i.e. $\text{features}(n_{t+1})$,  at each timestep~$t$}
 \KwResult{Location prediction $p_t$ for time $t+1$ at time $t$, at each timestep~$t$}
 
 \For{each timestep $t$}{
 
Compute the lead component $l(C_i^t )$ for each component $C_i^t$ 

$p_t = \arg\max_{i=1,\dots,10} w(n_{t+1},l(C_i^t ))$
}
 \caption{Finding the predicted location of the next terror attack}
\end{algorithm}

\subsection{Justification}
\label{subsec:Justification}
The design of prediction algorithm is motivated by the following aspects:
\begin{itemize}
\item The labels are taken into account by weighting the edges. This allows to completely ignore label signals on the graph and simplify the analysis.
\item The determination of one lead node per component allows to smoothen local variations inside a component, thus making the prediction algorithm more robust.
\item The choice of one lead component per component is justified by the fact that connected components are almost complete.
\end{itemize}

\subsection{Results}
\label{subsec:Results}

\begin{table}[H]
\caption{Prediction accuracy for different node distance weightings $f$}
\begin{center}
\begin{tabular}{l l l l}
\multicolumn{2}{l}{
\textbf{Weighting}}														& \textbf{Best skewness $\zeta$}		& \textbf{Accuracy}\\

Gaussian:			& $f(d)=e^{-d^2/\zeta}-e^{-1/\zeta}$							& \SI{0.01}{}						&\SI{50.5}{\percent} \\

Log-Exponential:	& $f(d)=e^{-d} \log\left( \frac{1+\zeta}{d+\zeta}\right)$				&\SI{0.1}{}							& \SI{50}{\percent} \\ 

Linear:			& $f(d)=1-d$											& N.A. 							&\SI{47}{\percent} \\

Square:			&$f(d)= \begin{cases}
1				&d < \zeta \\
0				& \text{otherwise}
				\end{cases}$											& \SI{0.1}{}						& \SI{43}{\percent}
\end{tabular}
\end{center}
\label{tab:Prediction accuracy for different node distance weightings}
\end{table}

\section{Conclusion}
\label{sec:Conclusion}
Subsection~\ref{subsec:terror attack quality} explains that the ``Terror Attacks" dataset contains flaws that make it difficult to analyze.

However, the results in Table~\ref{tab:results} show that predicting the location of an attack with its features is feasible even though the prediction is not very efficient. This result suggests that there is a link between the location of an attack and its characteristics (such as the type of the attack).

\bibliographystyle{ieeetr}
\bibliography{bib}

\end{document}  