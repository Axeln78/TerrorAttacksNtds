\section{Predicting Terror Attack Location}
\label{sec:Predictions}
The goal of this part is to predict the location of a terror attack based on its features and the history of previous attacks.
The algorithm used for prediction is the following:
\begin{enumerate}
\item From the dataset, select the 10 biggest connected components (``component'' in what follows).
\item Sort the dataset by date of terror attack.
\item At this point, a component represents a location, and the nodes are the terror attacks in chronological order.
\item Select one node per component that is strongly connected to the others, the ``lead" node.
\item Find the lead node $l^\star$ that is the most strongly linked to the new node (i.e. the next terror attack).
\item The predicted location of the next terror attack is the location of the component $l^\star$ belongs to.
\end{enumerate}

The determination of the lead node uses the features vector supplied with each node, and a weighting function $w$.
Let $w$ be the application that returns a weight for each pair of nodes $(n_1,n_2)$ in the graph $\mathcal{G}$, defined as
\begin{align}
w:~ \mathcal{G}^2	& \to \mathbb{R}^+ \\
(n_1,n_2) 			& \mapsto f(|n_1-n_2|)
\end{align}
where:
\begin{equation}
|n_1-n_2| = \| \text{features}(n_1)-\text{features}(n_2)\|_2 \\
\end{equation}
$\text{features}(n)$ is a binary features vector for each node $n\text{ in }\mathcal{G}$ and $f:~\mathbb{R}^+ \to  \mathbb{R}^+$ is the node distance weighting.
Examples for $f$ are given in Table~\ref{tab:Prediction accuracy for different node distance weightings} on page~\pageref{tab:Prediction accuracy for different node distance weightings}.

For each connected component, the lead node is determined as described below.

\vspace{1em}

\begin{algorithm}[H]
\vspace{-.5em}

 \KwData{Connected component $C$}
 
 \KwResult{Lead node $n_l$}
 
 Initialise $s(n)$ to zero. $s$ is a dictionary mapping a score $s(n)$ for each node $n$
 
 \For{each edge $e$ from $C$}{

Let $e=(n_1,n_2)$, $w$ be the weight of $e$

$s(n_1) \leftarrow s(n_1)+w$

$s(n_2) \leftarrow s(n_2)+w$

}
\KwRet{$n_l = \arg\max_{n \in C}s(n)$\vspace{.5em}}
 \label{alg:leadNode}
 \caption{Finding the lead node of a connected component with weighted edges}
\end{algorithm}

\vspace{1em}

Finally, the prediction algorithm is presented below. 

\vspace{1em}

\begin{algorithm}[H]
\vspace{-.5em}

 \KwData{Set of connected components $\{C_i^t\}$, $i=1,\dots,10$, and
 the features vector of the next terror attack $n_{t+1}$, i.e. $\text{features}(n_{t+1})$,  at each timestep~$t$}
 \KwResult{Location prediction $p_t$ at each timestep~$t$}
 
 \For{each timestep $t$}{
 
Compute the lead component $l(C_i^t )$ for each component $C_i^t$ 

$p_t = \arg\max_{i=1,\dots,10} w(n_{t+1},l(C_i^t ))$
}
 \caption{Finding the predicted location of the next terror attack}
\end{algorithm}

\subsection{Justification}
\label{subsec:Justification}
The design of prediction algorithm is motivated by the following aspects:
\begin{itemize}
\item The labels are taken into account by weighting the edges. This allows to completely ignore label signals on the graph and simplify the analysis.
\item The determination of one lead node per component allows to smoothen local variations inside a component, thus making the prediction algorithm more robust.
\item The choice of one lead component per component is justified by the fact that connected components are almost complete.
\end{itemize}

\subsection{Results}
\label{subsec:Results}

Running the algorithm aforementioned on the terror attacks dataset yields a prediction accuracy around $\frac{1}{2}$ (see Table~\ref{tab:Prediction accuracy for different node distance weightings} below), i.e. half of the predicted location is correct.

\begin{table}[H]
\caption{Prediction accuracy for different node distance weightings $f$}
\begin{center}
\begin{tabular}{l l l l}
\multicolumn{2}{l}{
\textbf{Weighting}}														& \textbf{Best skewness $\zeta$}		& \textbf{Accuracy}\\

Gaussian:			& $f(d)=e^{-d^2/\zeta}-e^{-1/\zeta}$							& \SI{0.01}{}						&\SI{50.5}{\percent} \\

Log-Exponential:	& $f(d)=e^{-d} \log\left( \frac{1+\zeta}{d+\zeta}\right)$				&\SI{0.1}{}							& \SI{50}{\percent} \\ 

Linear:			& $f(d)=1-d$											& N.A. 							&\SI{47}{\percent} \\

Square:			&$f(d)= \begin{cases}
1				&d < \zeta \\
0				& \text{otherwise}
				\end{cases}$											& \SI{0.1}{}						& \SI{43}{\percent}
\end{tabular}
\end{center}
\label{tab:Prediction accuracy for different node distance weightings}
\end{table}
